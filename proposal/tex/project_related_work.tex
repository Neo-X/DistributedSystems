
\section{Related Work}

\todo{Re-organize this and include some references}

To build an online real-time multiplayer game, there are two popular approaches known. One approach is called peer-to-peer lockstep, which is based on \ptoP architecture. In this approach, all clients start in same initial state and broadcast their every move. The overall performance of the system is dependent on slower clients in the system. Moreover, since the clients use broadcast methods to communicate, the approach has high message complexity.

To achieve better real-time support systems switched to a \clientServer model~\cite{DOOMfaq}. With this model the state of the game is stored on the server and clients send updates. This works well as latency is determined by the client to server connection (no slow client(s) halting the game). However, this was still too slow for real-time online games, which lead to the introduction of client-side prediction. Simply put client-side prediction allows the client to simulate its own version of the game (sending the results to the server) but the server can still step in and override the client's game state. This creates complexity in handling server overrides on clients smoothly (not just in code but also in animation and audio). Another reason for this model to gain traction was cheating could be preventing by validating all actions on the server.

The \ptoP model does handle fault tolerance more gracefully. Having the single server in the \clientServer model results in little fault tolerance.
