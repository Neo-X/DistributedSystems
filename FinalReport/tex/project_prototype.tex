\section{Prototype}
\label{sec:prototype}

	In this section, we describe the construction and design of the prototype system we developed. The system is written in Go-lang. For the purpose of testing, the client is designed to simulate an agent randomly moving and firing in a random direction periodically.
	We use a centralized \kvService in our system, which could be enhanced to distribute over the \localServer.
	
\subsection{Data Structures}

	The system uses a \emph{hashmap} to store the \gamestate where agent names are keys. The data for each agent is stored as value in the \gamestate in the form of a structure as shown Table~\ref{table:gamestate-struct}.


\begin{table}[htb]
	\centering
	\begin{tabular}{p{0.2\linewidth} | p{0.2\linewidth} | p{0.5\linewidth}}
		\textbf{Attribute name} & \textbf{Type} & \textbf{Description} \\ \hline
		Name & string & identifier for an agent\\ \hline
		Location & Vector & current location of the agent \\ \hline
		TimeStamp & int64  & vector clock for the agent \\  \hline
		LastUpdateTime & int64 & the last time the node has received a message from the agent\\ \hline
		Direction & Vector & the velocity of the agent\\ \hline	
	\end{tabular}
	\caption{\label{table:gamestate-struct} Outline of the data stored by each agent (\agentstate).}
\end{table}


\subsection{System Activation}

	To start the system, the \kvService and \activityServer are executed. The \activityServer initializes the keys in the \kvService that are used by the nodes to registered and active in the system. 
	
\subsection{Registration and Game State Construction}

	%For a new node to join the game, the node has to register itself in the \kvService. After registration the \localServer will send registration request to the \activityServer, which responds back with the node and agent identifier, along with the initial location of the agent in the game.

% \subsection{Game State Construction}
	For a new node to join the game, the node has to register itself in the \kvService. 
	The next step after registration is the \gameStateConstruction for the new node. For this to take place, the new joining node waits for a short period of time to receive location broadcasts from all, already present, nodes in the system. The location broadcasts contain the latest location of each agent in the game. Once the \gamestate is constructed, the newly joined node can begin simulating the game.	

\subsection{Move Event Processing}

	For our prototype, each client is responsible for a single agent. After every \gamestate update by the agent, the client sends a position update \move{\agent}{\position} to the \localServer. The \localServer, based on the position of agent in its \gamestate, verifies the validity of request. Any request that does not satisfy the following relation, will result in the request being ignored and the location agent overridden by the \localServer.


	\emph{if ( (distance/deltaTime) $>$ GameMaxVelocity) } \\
\begin{tabular}{l}
	\emph{where} \\ 
	\emph{distance := (new position of agent) $-$ (previous position of agent)}, \\
	\emph{clientDeltaTime := timeNow $-$ LastUpdateTime},  \\
	\emph{deltaTime := float64(clientDeltaTime)$/1000000000.0$}, \\
	\emph{GameMaxVelocity := predefined maximum velocity for an agent} \\
	\\
\end{tabular}	
	 
	If the request is valid, the \localServer broadcasts the received request to all other nodes in the system which update their own \gamestate as well as forward the update to their paired clients. The clients will update their \gamestate upon receipt of the message.

\subsection{Fire Event Processing}

	The execution of a \fire{\position}{\direction} command is very similar to the move action execution. The client sends a fire request to its \localServer. The \localServer, based on its own \gamestate, calculates if the shot hits any of the agents in the game. If it doesn't, the request is ignored. If the \localServer does find a possible shot, the \localServer forwards the request to the particular node paired with the client controlling that agent. Since that node is expected to have the most updated location of the agent being hit it is reasonable to have that node validate the event. The recipient node using its own \gamestate verifies the shot. If the \localServer finds the shot to be unsuccessful, the request is simply ignored. Whereas if the hit is successful, the recipient node broadcasts a \destroy{\agent_{j}} message to all the \localServers in the system and respawns the agent at a different location.
	
	Processing fire events on a different node is also advantageous in protecting against malicious clients. The event is first checked on the \localServer to verify that the location of the agent and target agent are correct. This potential shot is then sent to the node with the most up to date information on the target agent and verified again. The client for the target agent could try to avoid being shot by blocking incoming fire events to its \localServer. However, this would cause the game state the client and \localServer which will result in fire events from the client being invalidated by other nodes. We use this incentive-based method to discourage cheating with respect to fire events.