\section{Prototype}
\label{sec:prototype}

	In this section, we describe the construction and working of the prototype system we developed. The system is written in Go-lang. For the purpose of testing, the client is designed to simulate an agent randomly moving and firing around in random direction.
	In order to save time, we used centralized \kvService in our system, which can easily be enhanced to distribute over the \localServer.
	
\subsection{Data Structures}
	The system uses \emph{hashmap} to store the game state where node name act as key. The game state is stored as value in \emph{hashmap} in the form of structure as shown Table~\ref{table:gamestate-struct}.


\begin{table}[htb]
	\centering
	\begin{tabular}{p{0.2\linewidth} | p{0.2\linewidth} | p{0.6\linewidth}}
		\textbf{Attribute name} & \textbf{Type} & \textbf{Description} \\ \hline
		Name & string & identifier for an agent, alloted by \activityServer while joining the system \\ \hline
		Location & s3dm.V3 & current location of the agent \\ \hline
		TimeStamp & int64  & vector clock for the agent \\  \hline
		LastUpdateTime & int64 & the last time the node has received a message from the agent\\ \hline
		Direction & s3dm.V3 & the velocity of the agent with include magnitude and direction \\ \hline	
	\end{tabular}
	\caption{\label{table:gamestate-struct} Layout of an \agentstate.}
\end{table}


\subsection{Starting the System}
	To start the system, first of all the \kvService and \activityServer are started. The \activityServer initializes the keys in \kvService which are required by nodes to stay registered in the system. 
	
\subsection{Registeration}
	For a  new node to join the game, the node has to register itself in the system. The node registration is carried out through \activityServer. The \localServer at new joining node sends registration request to the \activityServer, which responds back with the node and agent identifier, along with the initial location of the agent in the game.

\subsection{Game State Construction}
	The next step after registeration is the \gameStateConstruction on the new joining node. For this to take place, the new joining node waits for a certain period of time to receive location broadcasts from all of the already present nodes in the system, which carries  the latest location of an agent. Once the \gamestate is constructed, the new joined node starts simulating the game.	

\subsection{Move action Execution}
	For our prototype, the clients simulate an agent randomly moving and firing around. After every frame is rendered in the game the client sends a position update request(\move{\agent}{\position}) to the \localServer.The \localServer based on position of agent in its \gamestate, verifies the validity of request. Any request that does not satisfy following relation, will have its location overridden by the \localServer.


	\emph{if (distance/deltaTime greater than GameMaxVelocity) } \\
\begin{tabular}{l}
	\emph{where} \\ 
	\emph{distance := (new position of agent) - (previous position of agent)}, \\
	\emph{deltaTime := float64(clientDeltaTime)/1000000000.0}, \\
	\emph{clientDeltaTime := timeNow - LastUpdateTime},  \\
	\emph{GameMaxVelocity := predefined maximum velocity for an agent} \\
	\\
\end{tabular}	
	 


	If the request is valid, the \localServer broadcasts the received request to all of the \localServers in the system which update their own \gamestate as well as forward it to their corressponding clients. The clients also update their \gamestates.

\subsection{Fire action Execution}
	The execution of \fire{\position}{\direction} command is very similar to the move action execution. The client sends a fire request to its \localServer. The \localServer, based on its own \gamestate, calculates if the shot hit any of the agents in the game. If it doesn't, the request is ignored. If it does, the \localServer forwards the request to that particular \localServer since that \localServer is expected to have the most updated location of the agent being hit. The recipient \localServer using its own \gamestate verifies the hit. If it finds the hit unsuccessful, the request is simply ignored. Whereas if the hit is successful, the recepient \localServer broadcasts a \destroy{\agent_{j}} message to all the \localServers in the system.