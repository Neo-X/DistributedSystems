
\section{Introduction}
\label{sec:Intro}

To build an online real-time multiplayer game, there are two popular methods.
One, is \ptoP \emph{lockstep}, which is based on the \ptoP architecture.
In this approach, all nodes start in the same initial state and each node broadcast every move to the all other nodes.
The overall performance of the system is dependent on slower nodes in the system.
Moreover, since the nodes use broadcasts to communicate, the method generates a significant volume of messages.

A second method is based on the client-server architecture, in which the game is also simulated on a server and each clients sends updates to the server.
The drawbacks of this method is that the server can become a bottleneck to the system performance and is a single point of failure.
From the point of view of support for this system there needs to be a large investment in infrastructure.
Also, due to additional game simulation load on the server, this approach is not scalable.

Cheating is serious problem in online games.
The gameplay needs to be fair in order to keep players interested in the game. 
Cheating can be done by players in the game sending information to other clients that should not be possible according to the current game state and invariants on the capabilities of the players in the game.
In the \ptoP architecture protecting against cheating is very difficult as clients send information directly to other clients.
The \clientServer method handles malicious players by simulating the game on an \emph{authoritative} server that verifies updates from clients.
In this work the concept of an authoritative server is extended to a distributed authoritative server.
Then, depending on the semantics of the game a particular authoritative server will verify propose \gamestate update.

We propose to construct a distributed \clientServer model in order to gracefully handle fail-stop scenarios.
We will use this system to support a simple computer game of a number of agents moving around in a 3D world called the \gamestate.
We assume no limits on bandwidth and have strong constraints on latency which significantly impacts online game experience~\cite{Claypool:2006:LPA:1167838.1167860}.

% Benefit of this solution:
% Having a more fault tolerant system using a distributed server. 
% Reducing latency by reducing by picking good servers
All online multiplayer systems are best-effort.
If we can create a system with the same responsiveness with fault tolerance then it is a success.

	

