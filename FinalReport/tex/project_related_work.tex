
\section{Overview of Technology}

To build an online real-time multiplayer game, there are two popular methods.
One, is \ptoP \emph{lockstep}, which is based on the \ptoP architecture.
A diagram of the \ptoP communication structure is shown in Figure~\ref{figure:p2p-vs-ClientServer}(a).
In this approach, all nodes start in the same initial state and each node broadcast every move to the all other nodes.
With nodes communicating directly the \gamestate can not advance until each node's move is received by every other node.
The overall latency of the system is then dependent on the slowest node in the system.
This system is also not tolerant to faulty nodes, as each node will wait and decide themselves if a node has failed.
Since the nodes use broadcasts to communicate, the method generates a significant volume of messages.

To achieve better real-time simulation, systems switched to a \clientServer model (see Figure~\ref{figure:p2p-vs-ClientServer}(b))~\cite{DOOMfaq}. 
With this model the \gamestate is stored on a server and clients send updates to the server.
This model reduces latency, for each client the latency is determined by the connection between that client and the server.
However, this model is still too slow for real-time online multiplayer games, which lead to the introduction of client-side prediction~\cite{bernier2001latency}.
Simply put, client-side prediction allows the client to simulate its own version of the game (sending the results to the server) but the server can still step in and override the client's \gamestate.
This creates complexity in handling server overrides on clients smoothly (not just in code but also in animation and audio).
Another reason for this model to gain traction was the ability to handle malicious clients by validating all actions on the server.


% The \ptoP model does handle fault tolerance more gracefully. Having the single server in the \clientServer model results in little fault tolerance.

\subsection{Multiplayer Network Design}

	There are two common networking structures for multiplayer gaming. The first is \ptoP, with the structure each client sends state updated to each other client in the game.
	The second, and more common, is the \clientServer design. In the \clientServer design each client sends updates to a single server and this server will relay these messages to the other clients playing the game. The communication structure for this design can be seen in Figure~\ref{figure:p2p-vs-ClientServer}(b).
	
\begin{figure}[ht]
	\centering
	\begin{tabular}{c c}
		Peer-to-Peer & Client-Server \\
		\includegraphics[width=0.48\linewidth]{../images/p2p-model-crop.pdf} &
		%trim=l b r t
		\includegraphics[width=0.48\linewidth]{../images/client-server-model-crop.pdf} \\
		(a) & (b)
	\end{tabular}

	\caption{\label{figure:p2p-vs-ClientServer} Two mutiplayer game networking models. The model on the left (a) is a \ptoP model where every client sends updates directly to ever other client in the game. The second model (b) is a \clientServer model. In this model all of the clients send updates to the server and the server send updates out to the clients.}
	\end{figure}
	
	We choose to base our work off of the \clientServer model for a number of reasons
	\begin{enumerate}
		\item The \clientServer model tends to have less latency
		\item The \clientServer model supports clients joining mid game
		\item The \clientServer model is less susceptible to cheating/malicious clients.
	\end{enumerate}
	
	As in most multiplayer game networking systems asynchronous communication is used. This is necessary to preserve the real-time nature of the game. Packet loss is considered not significant as a new packet with more up-to-date information will be sent soon after the lost packet.
	
